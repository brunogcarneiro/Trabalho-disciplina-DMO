%%% Section start %%%
\section{Introduction}
\label{sec-intro}
%
%Requirements prioritization is an important activity in the software life-cycle. During the development process, decisions about which ones are the most important requirements for a software system should be made. After that, during the operation of the system, decisions about how the software shall evolve and what path to take to guarantee the longevity of the software (now as a product) are as important as the ones made during design-time. Besides that, the quality of a software product is directly related to the ability of the software product satisfy the needs of the users by providing the desired features~\cite{sommerville2008engenharia}. Because of that, planning the development and also the evolution of a software, in order to specify the correct requirements and future updates are vital for the success of a software.As an example, we can mention World of Warcraft, the world's most successful massive multi-player online role playing game, had part of its success credited to the fact that the creators had over 12 years of updates and expansions precisely planned and prioritized. On the other side, bad decisions about the requirements implementation can lead to delays, cost extrapolation and in more serious cases, the failure of a software project~\cite{babar2011challenges}.
Software is becoming  pervasive in our daily activities,  from social activities, home management, to traveling and professional activities. On one side, as users we expect high quality software, on the other side,  as software producers, our main objective is to keep high  the quality of the software system we deliver, so to remain competitive in the market. This requires high quality development, maintenance and software evolution processes, which rest on continuous and effective decision-making to cope with technology changes and new  market's requests. 

%Software Systems are getting increasingly complex. 
For instance, for a software to be considered commercial, more requirements need to be implemented and higher quality standards need to be achieved. However, in most cases, because of limitations such as time and budget, not all functionalities can be implemented in the correct time, which contributes to the importance of well-grounded decisions about the path a software project should take during development. Besides, developing a software system is not an easy task on its own, since they usually are composed of many sub-systems, hardware components and even external software. Because of that, the decision-making process is vital for any kind of organization: decisions need to be made in strategical, tactical and even in operational levels of an organization~\cite{aurum2003fundamental}. Moreover, for organizations that deal with complex and time-consuming products, such as software systems or software-intensive products, this is specially true, since a bad decision can delay a project for months, extrapolate the cost budget and, in the most serious cases, cause the organization to go bankrupt. 

To cope with these problems, providing support and improving the decision making process, Papatheocharous et al.~\cite{papatheocharous2015decision,papatheocharous2018thegrade} propose the GRADE taxonomy, which intends to provide a structured instrument for collecting and documenting decision-making evidence. Its intent is to help software organizations in their projects. 

The GRADE taxonomy has been validated on decisions taken in several %29? 
industrial projects, providing evidences about its effectiveness. 
%Problem
%Nevertheless important weaknesses were pointed out by this validation, like the lack of  clear relationships between elements of the decision-making taxonomy, which makes it difficult for a new user 
%adopter of the taxonomy 
%to learn how to use it correctly.
%However, as a taxonomy, GRADE
For instance, the fact that this taxonomy provides a series of concepts (as categories) but does not specify the relations between theses categories, has been identified as a limitation, which can make it difficult for a new user to learn how to use it correctly.

As understanding the relationships between the concepts inherent to a domain are crucial part to understand the domain itself~\cite{guarino2015discussrelationship}, in this paper, we intend to fill this gap by providing an ontological analysis of the categories presented in the GRADE taxonomy, using the decision-making ontology, originally proposed in~\cite{guizzardi2018aligning}. Finally, it is important to mention that we chose the decision-making ontology to be the base of this ontological analysis because it is directly related to the domain of the taxonomy and because it is grounded in the Unified Foundational Ontology (UFO)~\cite{guizzardi:phdthesis05,guizzardi-et-al:ideas2008}. UFO is a widely accepted foundational ontology, providing a complete set of domain-independent concepts and their relations, that can be successfully used to provide an ontological interpretation and domain clarification.
%Because of these factors, plus the fact that a decision making process is a difficult task on its own that many researchers and international standards such as the IEEE Recommended practice for software requirements specifications~\cite{ieee830-1998} and the CMMI~\cite{team2010cmmi} take requirements prioritization as an important management activity for any organization.

%As Ruhe~\citeonline{ruhe2005decision} pointed out, requirements prioritization is known to be difficult because of the uncertainty and overall incompleteness of the information available. Moreover, there is a number of different techniques and frameworks presented in the scientific literature and even in international standards for requirements prioritization, each one with its own specificities, strength and weakness. Some methods are based on international standards while others are presented in the scientific literature and tested over years through case study, however, to the best of our knowledge no proposal takes into consideration the use of reference ontologies to support the decision making during the requirements prioritization process.

%This paper presents an work in progress of an ontological analysis of two requirements prioritization techniques presented in the scientific literature based on the decision making ontology~\cite{guizzardi2018aligning}


The rest of the paper is organized as follows, Section~\ref{sec-background} presents the GRADE taxonomy and the ontological foundations underlying the decision-making ontology. Section~\ref{sec-ontology} presents the ontological analysis of the GRADE taxonomy made based on the decision-making ontology. Section~\ref{sec-example} presents an illustrative example and a discussion about the added value of using the ontology. Section~\ref{sec-relatedwork} lists previous works about using ontology in Decision Making and ontological analysis of other decision-making methods. Section~\ref{sec-discussion} proposes a discussion 
%threat to validity and 
on the implication of this work and, finally, Section~\ref{sec-conclusions} concludes the paper.


