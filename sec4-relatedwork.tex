%%% Section start %%%
\section{Related Work}
\label{sec-relatedwork}
%[Anna] Related work (0,7 p.)
%Using ontology for decision making?
%Ontological analysis of other decision-making methods?

Mansingh and Rao~\cite{mansingh2014enhancing} present a method for enhancing the decision making process that is based in an organizational ontology. Authors argue that the ontology can be used to provide a better understanding of the domain in which the decision is being made. A series of steps are presented for the decision-maker to follow, these steps are organized in a way to focus on the understanding of the consequences of each alternative of the decision that is being made. Authors also suggest that a potential user of the method should create an that represents the domain of the decision.  In comparison with our ontological analysis and with the decision-making ontology, the term ontology here seems to be used in a loose way, since the paper does not provide a reference model of the referred ontology. The organizational ontology that is mentioned is presented in a taxonomy of core elements of the decision-making domain, such as goals, resources, tasks and roles, however, the relations between these elements are not properly explained. 

Chai and Liu~\cite{chai2010ontology} proposed a ontology-based framework focused on group decision process, called ONTOGDSS. Authors claim that the framework the management of the group decision-making process as a whole, from the evaluation of the alternatives to the aggregation of the decision, since the embedded ontology structure provides formal description features to facilitate the decision analysis. ONTOGDSS consists in two aspects: a group decision process, which should be followed and a hierarchical structure, which is composed by four layers, where the fourth one, called \textit{Ontology based Decision-resource layer} uses semantic extraction to represent and describe the decision problems that are stored in heterogeneous databases. In other words, ONTOGDSS defines a set of meta-data to describe attributes, objectives, constraints and criteria that are part of the decision process. Furthermore, with the properties and limitations of the decision problem represented, it can be analyzed accordingly, with tools like Ontobroker.
